\section{Einleitung}\label{sec:einleitung} % (fold)
Wer kennt’s nicht - Lust auf Cocktails, aber keine Ahnung, worauf genau? Keine Zeit oder Lust, erst
eine Einkaufsliste zu schreiben während ohnehin noch offene Flaschen im Schrank herum stehen? Statt
planlos neue Spirituosen zu kaufen, braucht es eine Lösung, die bequem leckere Rezepte liefert und
dabei vorhandene Reste sinnvoll verwertet.

Oder in anderen Worten: eine billige, nicht verschwenderische Möglichkeit, betrunken zu werden, ohne
Aufwand und ohne Frust.

Genau hier setzt unser Cocktail-Workflow an. Ziel ist es, mit minimalem Input schnell passende
Cocktailvorschläge zu erhalten. Dazu reicht die Anzahl der Personen, grobe Geschmacksvorlieben und
vor allem ein Foto der vorhandenen Spirituosen. Niemand hat Bock, jeden Alkoholbestand einzeln
abzutippen, deshalb übernimmt die App diese nervige Arbeit.

Die Funktionsweise lässt sich grob wie folgt zusammenfassen:

Nutzer geben Personenanzahl und Präferenzen an und laden Bilder der vorhandenen Getränke hoch.

Eine KI analysiert das Bild und erkennt die enthaltenen Spirituosen. Ein Prompt an ein KI-Modell
ermittelt darauf basierend geeignete Cocktails. Die App zeigt Vorschläge, inklusive fehlender
Zutaten an. Damit ist sofort klar, welche Produkte schon da sind und welche ggf. noch besorgt werden
müssen. Nach Auswahl eines Cocktails kann man zu einer passenden Einkaufsliste weitergeleitet
werden. Mit diesem Ansatz liefern wir einfache Cocktail-Rezepte mit Resteverwertung und das
ressourcenschonend, kostengünstig und so unkompliziert wie möglich.

In diesem Cookbook wird aufgezeigt, wie die Implementation einer solchen Toolchain technisch
realisierbar ist. Dabei werden konkrete Tools und Softwarekomponenten vorgestellt und beschrieben
wie sie konfiguriert werden.
% section Einleitung (end)
