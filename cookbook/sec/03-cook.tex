\section{Anleitung zur selbständigen Reproduktion}\label{sec:cook} % (fold)
Für die Reproduktion des Workflows mit identischer Einrichtung werden zwei Programme benötigt: nginx
als Reverse Proxy sowie Docker zur Ausführung des n8n-Containers, der Mongo-Datenbank und der
Einkaufslisten-Web-App.

Der \autoref{sub:einrichtung_nginx} kann beim localem Hosten übersprungen werden.

\subsection{Einrichtung nginx}\label{sub:einrichtung_nginx} % (fold)
Im bereitgestellten ZIP-Archiv befindet sich die Datei \verb|n8n.conf|. In dieser Datei ist in Zeile
4 und in Zeile 13 der korrekte Domainname einzutragen. In Zeile 10 und in Zeile 11 ist der Pfad zum
SSL-Zertifikat und zum zugehörigen privaten Schlüssel anzupassen. Nach der Anpassung ist die Datei
in das Verzeichnis \verb|/etc/nginx/sites-available| zu kopieren. Anschließend ist im Verzeichnis
\verb|/etc/nginx/sites-enabled| ein symbolischer Link auf die Konfigurationsdatei zu erstellen:

\begin{verbatim}
ln -s /etc/nginx/sites-available/n8n.conf /etc/nginx/sites-enabled/n8n.conf
\end{verbatim}

Dasselbe vorgehen muss mit der Datei \verb|shopping.conf| wiederholt werden.

Damit die Änderungen wirksam werden, ist der nginx-Dienst neu zu laden:

\begin{verbatim}
nginx -s reload
\end{verbatim}
% subsection Einrichtung nginx (end)

\subsection{n8n und Mongo}\label{sub:n_n_und_mongo} % (fold)
In der entpackten ZIP-Datei befindet sich die Datei \verb|.env.example|, die in \verb|.env|
umbenannt werden muss. Die dort gesetzten Werte sind für den Betrieb mit einer eigenen Domain und
HTTPS vorgesehen. Sollten die Services also auf einem Server unter einer eigenen Domain gehostet
werden, müssen diese Werte entsprechend auf die eigene Domain angepasst werden.

Wird das System hingegen lokal betrieben, ist der Domainname auf \verb|localhost:5678| und das
Protokoll auf \verb|http| zu ändern. Die Variable \verb|SHOPPING_URL| ist auf \\
\verb|http://localhost:3069| zu setzen. Unabhängig davon, ob lokal oder mit eigener Domain gehostet
wird, sollten die Passwörter in den Variablen \verb|MONGO_ROOT_PASS| und \\
\verb|MONGO_APP_PASS| stets angepasst werden.

Nach diesen Änderungen können n8n, MongoDB und die Einkaufslisten-Web-App mit dem Befehl
\verb|docker compose up -d| gestartet werden.

Standardmäßig wird dabei die aktuelle stabile Version von n8n verwendet. Die gewünschte Version kann
in der Datei \verb|docker-compose.yml| festgelegt werden. Die zuletzt getestete Version ist
\verb|1.105.3|.
% subsection n8n und Mongo (end)

\subsection{Workflow in n8n}\label{sub:workflow_in_n_n} % (fold)
Beim ersten Aufruf des n8n-Servers wird die Erstellung eines Benutzerkontos abgefragt. Nach
erfolgreichem Abschluss der Registrierung erfolgt die Weiterleitung zur Übersichtsseite, die in
\autoref{fig:n8n_overview} dargestellt ist.

In n8n selber müssen zunächst im \enquote{Credentials} Reiter zwei Zugangsdaten angelegt werden. Für
die MongoDB müssen in den Feldern \enquote{Database}, \enquote{User} und \enquote{Password} die Daten
eingetragen werden, die in der \verb|.env| Datei als \verb|MONGO_APP_DB|, \verb|MONGO_APP_USER| und
\verb|MONGO_APP_PASS| festgelegt wurden (siehe \autoref{fig:n8n_mongo_creds}).

Für die KI muss noch ein API Schlüssel von einem OpenAI Benutzerkonto hinterlegt werden. Dazu wird
das zweite Zugangsdatum erstellt (siehe \autoref{fig:n8n_openai_creds}).

Nachdem die beiden Zugangsdaten angelegt wurden, kann ein neuer Workflow erstellt werden. Im
Bearbeitungsmodus angekommen, befindet sich oben rechts im Dreipunktemenü die Auswahl ein
vorhandenen Workflow von einer Datei zu importieren. Im ZIP-Archiv ist der Workflow unter dem Namen
\verb|Promillo.json| zu finden (siehe \autoref{fig:n8n_import}).

Die Nodes, die Zugangsdaten benötigen werden rot makiert, diese müssen mit Doppelklick einmal
geöffnet und in der oberen linken Ecke geschlossen werden. Die Zugangsdaten verwenden zwar den
standard Namen, werden von n8n aber mit eindeutigen IDs referenziert. Aufgrund des anlegens der
Zugangsdaten werden im Workflow nicht die richtigen IDs referenziert, durch das Öffnen und Schließen
werden diese jedoch korriegiert.

Zum Schluss muss der Workflow in der oberen rechten Ecke gespeichert und aktiviert werden. Der
Workflow ist damit unter der eigenen Domaine mit dem Pfad \verb|/form/alcohol| erreichbar.
% subsection Workflow in n8n (end)
% section Cook (end)
