\section{Technische Beschreibung}\label{sec:descr} % (fold)

\subsection{Blackbox-Sicht}\label{sec:descr:blackbox}
[Ich warte auf Ole, damit wir nicht Sachen doppelt schreiben]

\subsection{Komponenten}\label{sec:descr:components}

Der Workflow ist aus mehreren gedanklichen Komponenten zusammengesetzt, welche wiederum aus kleineren Teilen bestehen. \\

\textbf{Nutzerinformationen} \\
Damit der Workflow laufen und eine Antwort liefern kann, muss der Nutzer relevante Informationen bereitstellen.
Diese werden über ein Formular abgefragt. Die Fragen sind:
\begin{itemize}
    \item Wie viele Personen wollen trinken?
    \item Welche Geschmacks-Präferenzen haben die Personen?
    \item Bilder über zur Verfügung stehenden Getränken oder anderen Lebensmitteln.
\end{itemize}

Sollte eine Personenanzahl < 1 eingegeben werden, wird die Anzahl automatisch auf 1 gesetzt.
Anschließend kann der Nutzer auf den Button klicken, damit der Workflow gestartet wird.
Im Anschluss wird der Nutzer gebeten zu warten, bis die nächste Komponente abgeschlossen ist. \\

\textbf{Bildanalyse} \\
Die Bilder werden an ein KI-Modell übergeben. Das Modell versucht das im Vordergrund stehende
Getränk zu erkennen. Dabei wird der Name des Getränkes, sowie das Fassungsvolumen in Milliliter und
der aktuellen Füllstand in Prozent ermittelt. Wir nutzen dafür das Modell \textbf{GPT-4O-MINI}
welches wir mit dem Workflow verbunden haben. \\

Der genutzte Prompt sieht wie folgt aus:
\begin{verbatim}
Erkenne auf jedem Bild genau eine Flasche (die im Fokus bzw. im Vordergrund steht) und gib folgende Angaben im exakt angegebenen Format aus:

{Genaue Bezeichnung des Getränks,z. B. “Coca-Cola Zero Sugar”,“Gerolsteiner Mineralwasser Medium”,“Volvic Mango”},{Maximalvolumen in ml},{Füllstand in %}
\end{verbatim}

Die Ausgabe des Modells erfolgt im JSON-Format, für die weitere Nutzung der Daten im Workflow. \\

\textbf{Datenvalidierung} \\
Zur korrekten Weiterverarbeitung der Daten wird der Nutzer gebeten, diese zu validieren bzw. zu
korrigieren. Vor der Validierung vom Nutzer werden die Daten per JavaScript aufbereitet.

Zum genaueren Verständnis kann der folgende Code in \autoref{lst:descr:js:code1} betrachtet werden.
Es ist zu beachten, dass teilweise n8n Spezifische Syntax verwendet wird.

Anschließend werden dem Nutzer sämtliche Daten im Formular angezeigt, welche er bei bedarf anpassen
bzw. überschreiben kann. Wenn keine Veränderungen vorgenomen werden, wird der Wert vom KI-Modell
übernommen.

Daraufhin werden die Daten wieder per JavaScript verarbeitet, damit sie im nächsten Schritt von
einem weiteren KI-Modell genutzt werden können. \autoref{lst:descr:js:code2} zeigt den Code, welcher
dafür genutzt wird. \\

\textbf{Rezeptvorschlag} \\
Die validierten Daten, welche aus der vorherigen Komponente stammen, werden nun an ein weiteres
KI-Modell gegeben. Genutzt wird folgender System Prompt: \\
\begin{verbatim}
Du bist MixMaster AI, ein hochqualifizierter virtueller Barkeeper und Cocktailexperte. Deine Aufgabe ist es, auf Basis der vom Nutzer angegebenen Präferenzen, der vorhandenen Zutaten und der gewünschten Anzahl an Personen passende Cocktailrezepte vorzuschlagen. Nutze dazu die Ressource „CocktailDB“ – eine Datenbank mit bekannten Cocktailrezepten als Tool. Beachte dabei:

1. **Eingaben**
   - Präferenzen (z. B. Geschmack: süß, sauer, herb etc.)
   - Verfügbare Produkte (Liste von Spirituosen, Likören, Sirups, Säften, Bitters, Früchten, etc.)
   - Anzahl Personen, für die gemixt werden soll

2. **Verarbeitung**
   - Greife auf CocktailDB zu, um passende Rezepte zu finden.
   - Nutze nicht zwingend alle verfügbaren Zutaten – wähle die besten Kombinationen passend zu den Präferenzen.
   - Falls ein Rezept eine oder mehrere Zutaten enthält, die der Nutzer nicht vorrätig hat, führe diese als „fehlende Zutaten“ gesondert auf.

Verhalte dich stets professionell, freundlich und präzise.
\end{verbatim}

und folgender User Prompt:
\begin{verbatim}
Erstelle 5 kreative aber auch gängige Cocktail-Vorschläge auf Basis der folgenden Angaben:

**Präferenzen**: {{ $('Edit Fields').item.json.preferenzen }}
**Anzahl Personen**: {{ $('Edit Fields').item.json.personen }}
**Verfügbare Produkte**: {{ $json.products }}

Die Inhalte von Verfügbare Produkte sind {Produkt},{Maximales Volumen in Milliliter},{Aktueller Inhalt in Prozent}
\end{verbatim}

Als Modell wird erneut \textbf{GPT-4O-MINI} genutzt.
Zudem kommt eine verbundene Mongo-Datenbank zum Einsatz. Darin können Rezepte gespeichert werden,
auf welche das KI-Modell zugreifen kann. Zudem werden diese Rezepte vom Modell bevorzugt ausgegeben,
wenn sie mit den vermittelten Daten eine Schnittmenge bilden. Dadurch können Präferenzen des Nutzers
besser berücksichtigt werden, als auch die Halluzinationen des Modells reduziert werden.

Die Ausgabe des Modells erfolgt im JSON-Format, in folgendem Schema: \autoref{lst:descr:js:code3}.

\textbf{Ergebnispräsentation} \\
Die Ergebnisse der vorherigen Komponente werden dem Nutzer präsentiert. Dafür werden die Daten
erneut in JavaScript aufbereitet, damit sie in HTML Code umgewandelt werden. Der
\autoref{lst:descr:js:code4} zeigt den Code, welcher dafür genutzt wird.

Anschließend wird der übergebene HTML Code wieder im Formular angezeigt.

Dazu kommen noch folgende CSS Styles, um die Darstellung zu verbessern: \autoref{lst:descr:js:code5}.

\textbf{Einkaufslisten-API} \\
Um fehlende Zutaten in eine Einkaufsliste zu übertragen, werden diese per POST-Request an eine dafür
angepasste Web-App übermittelt. Dafür wird pro Getränk eine eindeutig identifizierbare Zeichenfolge
generiert. In der Web-App werden anschließend für alle Getränke eigene Einkaufslisten
bereitgestellt, welche über die Zeichenfolgen erreichbar sind.
